\chapter{Elaboració d'atacs per a la generació de trànsit maliciós}

En aquest capítol es presenten els atacs que s'han implementat per a generar trànsit maliciós en l'entorn IoMT descrit a \ref{sec:Topologia}. Aquests atacs tenen com a objectiu comprometre la seguretat dels dispositius connectats o del broker intentant ser fidels a la realitat d'un escenari real. 

S'han utilitzat les eines descrites a \ref{sec:Eines} així com diversos blocs de codi elaborat en Python i funcionalitats pròpies de Linux. 


Per tal de generar dades que poden ser fetes servir per a l'entrenament de models de detecció d'intrusions basats en \textit{machine learning}.



\section{Atacs de descobriment de dispositius}
\label{sec:Recon}

El primer atac que se sol dur a terme en un entorn IoT és el descobriment de dispositius. Aquest se centra en identificar els dispositius connectats a la xarxa i obtenir informació sobre ells, com ara adreces IP, ports oberts i serveis disponibles. Això permet als atacants conèixer la topologia de la xarxa i identificar possibles vulnerabilitats.

Primer de tot, s'ha implementat un atac per aconseguir descobrir l'adreça IP del broker MQTT, ja que és l'element central de la comunicació entre els dispositius IoMT. Per a això, s'ha utilitzat l'eina \textit{nmap} \ref{sec:NMAP} per escanejar la xarxa sencera. Mitjançant Nmap, s'ha aplicat la detecció de ports oberts 1883 i 8883 (en el cas d'ús de TLS), ja que el broker MQTT, per rebre connexions dels clients ha de tenir aquests ports oberts.

També, una estratègia interessant és la detecció de serveis mitjançant l'opció -sV en Namp. Això ens permet descobrir si en el port 1883 o 8883 escanejat, s'està executant un servei de MQTT com per exemple Mosqitto 3.1.1.

Una execució utilitzada és:
\begin{tcblisting}{colback=white, colframe=black!70, listing only}
    nmap -sV -p 1883,8883 192.168.0.0/24
\end{tcblisting}

Aquest atac realitzarà una gran quantitat de peticions ARP a totes les IP del rang especificat i, en cas que contestin intentarà iniciar una connexió TCP als ports especificats (1883 i 8883), recopilant la informació de les respostes. Seguidament, mitjançant scripts propis de Nmap amb diverses peticions TCP intentarà esbrinar quins serveis s'estan executant.  

\textbf{capt wireshark}

Però aquesta execució no és del tot realista, ja que és fàcilment detectable. Per això per a poder generar trànsit maliciós de forma més realista, s'han d'utilitzar estratègies menys sorolloses, és a dir on es generi menys trànsit i sigui menys detectable.

Una opció és utilitzar inicialment un escaneig general arp o icmp més lent per tal de recopilar les IPs actives, i llavors fer un escaneig més específic a les IPs que han respost. Així, es redueix el trànsit generat i es fa més difícil la detecció de l'atac.

Per a fer el primer escaneig més silenciós es pot utilitzar l'opciṕo -T per ajustar la velocitat (un valor -T2 per una xarxa /24 pot trigar uns 10 minuts). També podem afegir l'opció -n per evitar la resolució de noms DNS, ja que al ser una xarxa local no és necessari. 

Llavors, per a fer el segon escaneig més específic, es pot utilitzar l'opció -Pn per evitar el ping (ja que ja sabem que el dispositiu està actiu gràcies al primer escaneig). També es pot afegir -sS per evitar completar el handshake TCP i així ser menys detectable. Ara podem afegir els ports específics i la detecció de serveis. Un exemple seria: 
\begin{tcblisting}{colback=white, colframe=black!70, listing only}
    nmap -sn -n -T2 192.168.0.0/24   #primer escaneig.
    nmap -Pn -sV -sS -p 1883,8883 IPs-actives.txt  #segon escaneig amb la llista d'IPs actives obtinguda del primer escaneig.
\end{tcblisting}

\textbf{capt wireshark}

Per a fer l'esaneig més ràpid, també he utilitzat masscan per a la detecció del broker, però no és tant eficient i és menys sigilós.

\begin{tcblisting}{colback=white, colframe=black!70, listing only}
    masscan 192.168.0.0/24 -p1883,8883 --rate 1000 -oG masscan.txt
\end{tcblisting}

\textbf{capt wireshark?}

Per a la detecció dels clients IoMT, amb el primer escaneig ja veiem els dispositius actius a la xarxa, però no podem saber si realment són dispositius IoMT, per això podem utilitzar la detecció de \textit{fingerprint} dels dispositius, referit a extreure tota la informació possible que descriu i diferència un dispositiu. 

El fingerprint es pot aconseguir mitjançant l'opció \textit{-o} de Nmap que detecta el sistema operatiu o també scripts NSE com \textit{"banner"} o \textit{"fingerprint-settings"}, encara que, són mètodes molt intrusius i, per tant, molt detectables. També es pot optar per a utilitzar datasets públics d'adreces MAC conegudes de fabricants com ara Philips Health, Siemens Healthcare o també Raspberry Pi i altres marques dispositius embeguts que siguin usualment utilitzats com a clients IoMT.

\begin{table}[H]
\centering
\begin{tabular}{@{}lll@{}}
\toprule
\textbf{Registry} & \textbf{Assignment} & \textbf{Organization Name} \\
\midrule
MA-S & 70B3D5B91 & Cardinal Health \\
MA-S & 8C1F6448D & Health Care Originals, Inc. \\
MA-S & 8C1F64C79 & Hills Health Solutions \\
MA-S & 8C1F6430C & Hills Health Solutions \\
MA-S & 001BC50B0 & Miracle Healthcare, Inc. \\
MA-S & 70B3D5FC4 & PHYZHON Health Inc \\
MA-S & 8C1F64B64 & Sensus Healthcare \\
MA-S & 8C1F64804 & Siemens Healthcare Diagnostics \\
MA-S & 8C1F6454A & Sound Health Systems \\
\bottomrule
\end{tabular}
\caption{Petit extret del fitxer "MAC Address Block Large (MA-S)" de \cite{iotmaclist} filtrat per a visualitzar només fabricants IoMT on es veuen parelles IniciMAC - Fabricant}  % Peu de figura
\label{fig:iot_manufacturers}
\end{table}
