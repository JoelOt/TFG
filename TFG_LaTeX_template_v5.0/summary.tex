% DON'T CHANGE THIS LINE
\addcontentsline{toc}{section}{\ifcase\doclanguage\or Resum \or Resumen \else Summary\fi}

%%%% PLEASE REPLACE TEXTS WITH YOUR OWN CONTENT %%%%

%%% RESUM EN CATALÀ
\begin{center}
  \huge\bfseries\raggedright Resum~\hrulefill
\end{center}
Els dispositius d'internet de les coses en l'entorn mèdic (IoMT) estan en clar creixement i adopció i es preveu que en els propers anys la seva presència sigui encara més gran. Això implica que la seguretat d'aquests dispositius és un tema de gran importància, ja que per la seva naturalesa de baixos recursos i criticitat de les dades amb les que tracten, poden ser molt vulnerables ciberatacs que comprometen la privacitat i la seguretat dels pacients. 

Aquest treball de fi de grau té com a objectiu principal l'estudi i la generació de ciberatacs sobre entorns mèdics basats en dispositius IoMT, especialment utilitzant el protocol "Message Queuing Telemetry Transport" (MQTT), amb l'objectiu d'exposar vulnerabilitats específiques dels sistemes IoMT i aportar una eina que permeti automatitzar la creació i recopilació de trànsit maliciós per a l'entrenament de sistemes de detecció d'intrusions basats en machine learning.%%% RESUMEN EN CASTELLANO
\begin{center}
  \huge\bfseries\raggedleft\vspace*{.5\baselineskip} \hrulefill ~Resumen
\end{center}
Los dispositivos del Internet de las Cosas en el entorno médico (IoMT) están en claro crecimiento y adopción, y se prevé que en los próximos años su presencia sea aún mayor. Esto implica que la seguridad de estos dispositivos es un tema de gran importancia, ya que, por su naturaleza de bajos recursos y la criticidad de los datos con los que trabajan, pueden ser muy vulnerables a ciberataques que comprometan la privacidad y la seguridad de los pacientes.

Este trabajo de fin de grado tiene como objetivo principal el estudio y la generación de ciberataques sobre entornos médicos basados en dispositivos IoMT, especialmente utilizando el protocolo "Message Queuing Telemetry Transport" (MQTT), con el objetivo de exponer vulnerabilidades específicas de los sistemas IoMT y aportar una herramienta que permita automatizar la creación y recopilación de tráfico malicioso para el entrenamiento de sistemas de detección de intrusiones basados en aprendizaje automático.
%%% ENGLISH SUMMARY
\begin{center}
  \huge\bfseries\raggedright\vspace*{.5\baselineskip} Summary~\hrulefill
\end{center}
Internet of Things devices in medical environments (IoMT) are experiencing clear growth and adoption, and their presence is expected to increase even further in the coming years. This means that the security of these devices is of great importance, as their low-resource nature and the criticality of the data they handle make them highly vulnerable to cyberattacks that can compromise patient privacy and safety.

This bachelor's thesis aims primarily to study and generate cyberattacks on medical environments based on IoMT devices, especially using the Message Queuing Telemetry Transport (MQTT) protocol, with the goal of exposing specific vulnerabilities in IoMT systems and providing a tool that automates the creation and collection of malicious traffic for the training of intrusion detection systems based on machine learning.